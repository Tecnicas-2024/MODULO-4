\documentclass{article}

\usepackage{arxiv}

\usepackage[utf8]{inputenc} % allow utf-8 input
\usepackage[T1]{fontenc}    % use 8-bit T1 fonts
\usepackage{lmodern}        % https://github.com/rstudio/rticles/issues/343
\usepackage{hyperref}       % hyperlinks
\usepackage{url}            % simple URL typesetting
\usepackage{booktabs}       % professional-quality tables
\usepackage{amsfonts}       % blackboard math symbols
\usepackage{nicefrac}       % compact symbols for 1/2, etc.
\usepackage{microtype}      % microtypography
\usepackage{graphicx}

\title{Infraestructuras críticas-Sistemas de Transporte}

\author{
    Futuros ingenieros
    \thanks{Técnicas y Herramientas Modernas I-Infraestructuras}
   \\
    Universidad Nacional de Cuyo \\
  Ingenieros J. 1400 \\
  \texttt{ingenieria.uncuyo.edu.ar} \\
  }


% tightlist command for lists without linebreak
\providecommand{\tightlist}{%
  \setlength{\itemsep}{0pt}\setlength{\parskip}{0pt}}


% Pandoc citation processing
\newlength{\cslhangindent}
\setlength{\cslhangindent}{1.5em}
\newlength{\csllabelwidth}
\setlength{\csllabelwidth}{3em}
\newlength{\cslentryspacingunit} % times entry-spacing
\setlength{\cslentryspacingunit}{\parskip}
% for Pandoc 2.8 to 2.10.1
\newenvironment{cslreferences}%
  {}%
  {\par}
% For Pandoc 2.11+
\newenvironment{CSLReferences}[2] % #1 hanging-ident, #2 entry spacing
 {% don't indent paragraphs
  \setlength{\parindent}{0pt}
  % turn on hanging indent if param 1 is 1
  \ifodd #1
  \let\oldpar\par
  \def\par{\hangindent=\cslhangindent\oldpar}
  \fi
  % set entry spacing
  \setlength{\parskip}{#2\cslentryspacingunit}
 }%
 {}
\usepackage{calc}
\newcommand{\CSLBlock}[1]{#1\hfill\break}
\newcommand{\CSLLeftMargin}[1]{\parbox[t]{\csllabelwidth}{#1}}
\newcommand{\CSLRightInline}[1]{\parbox[t]{\linewidth - \csllabelwidth}{#1}\break}
\newcommand{\CSLIndent}[1]{\hspace{\cslhangindent}#1}

\begin{document}
\maketitle


\begin{abstract}
Este paper busca demostrar como la infraestructura crítica juega un
papel fundamental en el funcionamiento y bienestar de las sociedades.
\end{abstract}


\pagebreak

\hypertarget{introducciuxf3n}{%
\section{Introducción}\label{introducciuxf3n}}

En el mundo globalizado actual, la infraestructura crítica juega un
papel fundamental en el funcionamiento y bienestar de las sociedades. El
sistema de transporte, como componente vital de esta infraestructura, es
esencial para la economía global, facilitando el movimiento de bienes,
servicios y personas. El Canal de Panamá, una de las infraestructuras de
transporte más importantes del mundo, enfrenta diversos desafíos que
amenazan su eficiencia y seguridad. Este trabajo se centra en analizar
la infraestructura crítica del sistema de transporte, con un enfoque en
el Canal de Panamá, y propone el uso de gemelos digitales como una
solución innovadora para mejorar su operación y resiliencia.

\hypertarget{quuxe9-es-una-infraestructura-cruxedtica}{%
\subsection{¿Qué es una infraestructura
crítica?}\label{quuxe9-es-una-infraestructura-cruxedtica}}

Son todos aquellos sistemas físicos o virtuales que hacen posible las
funciones y servicios considerados esenciales y que contribuyen al buen
desempeño de los sistemas más básicos a nivel social, económico,
medioambiental y político. Cualquier alteración o interrupción en su
suministro, debido a causas naturales (una catástrofe climática, por
ejemplo) o provocada por el factor humano (como un ataque cibernético a
una central de energía eléctrica) podría acarrear graves consecuencias.

\includegraphics{1I.png}

\hypertarget{tipos-de-infraestructura-cruxedtica}{%
\subsection{Tipos de infraestructura
crítica}\label{tipos-de-infraestructura-cruxedtica}}

Nos brinda una visión general de las principales infraestructuras
críticas ({``Infraestructuras Críticas: Definición, Planes, Riesgos,
Amenazas y Legislación''} n.d.)

Estas se agrupan comúnmente en sectores o áreas estratégicas, cada una
con sus propias características y desafíos específicos. A continuación
se presentan los principales tipos de infraestructuras críticas:

\hypertarget{sector-de-quuxedmicos}{%
\paragraph{\texorpdfstring{1. \textbf{Sector de
Químicos}}{1. Sector de Químicos}}\label{sector-de-quuxedmicos}}

Incluye la producción, almacenamiento y transporte de productos
químicos, los cuales son esenciales para diversas industrias pero
también representan un riesgo significativo si no se manejan
adecuadamente

\hypertarget{sector-de-comunicaciones}{%
\paragraph{\texorpdfstring{2. \textbf{Sector de
Comunicaciones}}{2. Sector de Comunicaciones}}\label{sector-de-comunicaciones}}

Comprende las telecomunicaciones, redes de datos y otros servicios de
comunicación que son fundamentales para la conectividad y el intercambio
de información en tiempo real.

\hypertarget{sector-de-energuxeda}{%
\paragraph{\texorpdfstring{3. \textbf{Sector de
Energía}}{3. Sector de Energía}}\label{sector-de-energuxeda}}

Abarca la generación, transmisión y distribución de electricidad, así
como las industrias de petróleo y gas, que son esenciales para casi
todas las actividades económicas y sociales.

\hypertarget{sector-de-servicios-de-emergencia}{%
\paragraph{\texorpdfstring{4. \textbf{Sector de Servicios de
Emergencia}}{4. Sector de Servicios de Emergencia}}\label{sector-de-servicios-de-emergencia}}

Incluye las agencias de policía, bomberos, servicios médicos de
emergencia y otras entidades de respuesta a emergencias que son
cruciales para la seguridad y el bienestar de la población.

\hypertarget{sector-de-presas}{%
\paragraph{\texorpdfstring{5. \textbf{Sector de
Presas}}{5. Sector de Presas}}\label{sector-de-presas}}

Se refiere a las estructuras y sistemas asociados con presas y embalses,
que son vitales para el suministro de agua, control de inundaciones y
generación de energía hidroeléctrica.

\hypertarget{sector-de-defensa-industrial}{%
\paragraph{\texorpdfstring{6. \textbf{Sector de Defensa
Industrial}}{6. Sector de Defensa Industrial}}\label{sector-de-defensa-industrial}}

Involucra la producción y mantenimiento de armas, equipos y tecnologías
militares necesarias para la defensa nacional.

\hypertarget{sector-de-alimentaciuxf3n-y-agricultura}{%
\paragraph{\texorpdfstring{7. \textbf{Sector de Alimentación y
Agricultura}}{7. Sector de Alimentación y Agricultura}}\label{sector-de-alimentaciuxf3n-y-agricultura}}

Cubre la producción, procesamiento y distribución de alimentos y
productos agrícolas, asegurando el suministro continuo de alimentos a la
población.

\hypertarget{sector-de-gobierno-y-servicios-puxfablicos}{%
\paragraph{\texorpdfstring{8. \textbf{Sector de Gobierno y Servicios
Públicos}}{8. Sector de Gobierno y Servicios Públicos}}\label{sector-de-gobierno-y-servicios-puxfablicos}}

Incluye las instalaciones y servicios esenciales del gobierno, que son
cruciales para el funcionamiento administrativo y operativo del país.

\hypertarget{sector-financiero}{%
\paragraph{\texorpdfstring{9. \textbf{Sector
Financiero}}{9. Sector Financiero}}\label{sector-financiero}}

Abarca los bancos, mercados financieros y otras instituciones
financieras, que son fundamentales para la estabilidad económica y el
crecimiento.

\hypertarget{sector-de-salud-y-salud-puxfablica}{%
\paragraph{\texorpdfstring{10. \textbf{Sector de Salud y Salud
Pública}}{10. Sector de Salud y Salud Pública}}\label{sector-de-salud-y-salud-puxfablica}}

Comprende hospitales, clínicas, laboratorios y otros servicios y
recursos de salud pública, esenciales para la atención médica y el
bienestar de la población.

\hypertarget{sector-de-tecnologuxedas-de-la-informaciuxf3n}{%
\paragraph{\texorpdfstring{11. \textbf{Sector de Tecnologías de la
Información}}{11. Sector de Tecnologías de la Información}}\label{sector-de-tecnologuxedas-de-la-informaciuxf3n}}

Incluye los sistemas y servicios de tecnologías de la información como
hardware, software y redes, que son vitales para la comunicación y la
gestión de datos.

\hypertarget{sector-nuclear}{%
\paragraph{\texorpdfstring{12. \textbf{Sector
Nuclear}}{12. Sector Nuclear}}\label{sector-nuclear}}

Se refiere a las instalaciones y actividades relacionadas con la energía
nuclear y materiales radiactivos, que requieren un manejo y protección
rigurosos.

\hypertarget{sector-de-transporte}{%
\paragraph{\texorpdfstring{13. \textbf{Sector de
Transporte}}{13. Sector de Transporte}}\label{sector-de-transporte}}

Incluye sistemas de transporte aéreo, terrestre, marítimo y ferroviario,
que son esenciales para el movimiento de personas y bienes.

\hypertarget{sector-de-agua-y-aguas-residuales}{%
\paragraph{\texorpdfstring{14. \textbf{Sector de Agua y Aguas
Residuales}}{14. Sector de Agua y Aguas Residuales}}\label{sector-de-agua-y-aguas-residuales}}

Abarca la gestión del suministro de agua potable y el tratamiento de
aguas residuales, cruciales para la salud pública y el medio ambiente.

\hypertarget{sector-de-instalaciones-cruxedticas}{%
\paragraph{\texorpdfstring{15. \textbf{Sector de Instalaciones
Críticas}}{15. Sector de Instalaciones Críticas}}\label{sector-de-instalaciones-cruxedticas}}

Comprende las instalaciones comerciales, residenciales, de
entretenimiento y de servicios que tienen un impacto significativo en la
seguridad y economía.

\hypertarget{sector-de-materiales-cruxedticos-y-manufactura}{%
\paragraph{\texorpdfstring{16. \textbf{Sector de Materiales Críticos y
Manufactura}}{16. Sector de Materiales Críticos y Manufactura}}\label{sector-de-materiales-cruxedticos-y-manufactura}}

Involucra la producción de materiales esenciales y bienes manufacturados
cruciales para la seguridad y economía del país.

\hypertarget{sector-de-sistemas-de-transporte-como-infraestructura-cruxedtica}{%
\subsection{Sector de sistemas de transporte como infraestructura
crítica}\label{sector-de-sistemas-de-transporte-como-infraestructura-cruxedtica}}

El sistema de transporte incluye la aviación, la red de autopistas y
autovías, el transporte marítimo, el transporte público y ferroviario de
pasajeros, los sistemas de oleoductos, el transporte ferroviario de
mercancías, y los servicios de correos y envíos Este se considera una
infraestructura crítica porque es esencial para el funcionamiento y la
seguridad de una sociedad. A continuación, ({``Critical {Infrastructure}
{Sectors} {\textbar} {CISA}''} n.d.) detalla los componentes principales
y las razones clave:

\begin{itemize}
\tightlist
\item
  \textbf{Movilidad y conectividad:}
\end{itemize}

Un sistema de transporte seguro y eficiente es crucial para la seguridad
nacional. Permite la rápida movilización de recursos en caso de
emergencias y desastres

\begin{itemize}
\tightlist
\item
  \textbf{Impacto económico:}
\end{itemize}

El transporte es vital para el comercio y la industria. La interrupción
de los sistemas de transporte puede causar pérdidas económicas
significativas y afectar la cadena de suministro.

\begin{itemize}
\tightlist
\item
  \textbf{Servicios esenciales:}
\end{itemize}

Muchas otras infraestructuras críticas, como la salud, la energía y las
telecomunicaciones, dependen del transporte para su funcionamiento. Por
ejemplo, los hospitales necesitan suministros médicos que se transportan
por carretera, aire o ferrocarril.

\begin{itemize}
\tightlist
\item
  \textbf{Respuesta a emergencias:}
\end{itemize}

En situaciones de emergencia, como desastres naturales o ataques
terroristas, el transporte es crucial para la evacuación y el suministro
de ayuda humanitaria.

\hypertarget{desarrollo}{%
\section{Desarrollo}\label{desarrollo}}

\hypertarget{caso-de-estudio-el-canal-de-panamuxe1}{%
\section{Caso de estudio: el canal de
Panamá}\label{caso-de-estudio-el-canal-de-panamuxe1}}

\hypertarget{funcionamiento}{%
\subsection{Funcionamiento}\label{funcionamiento}}

El Canal de Panamá es una obra de ingeniería que permite a los barcos
transitar entre el Océano Atlántico y el Océano Pacífico.

Su funcionamiento se basa en un sistema de esclusas y compuertas que
controlan el flujo de agua para elevar y descender los barcos a través
de diferentes niveles del canal.

Detalla ({``Descubre {Cómo} {Funciona} El {Canal} de {Panamá}:
{Historia} y {Funcionamiento}''} 2023)

\includegraphics{2I.png}

\hypertarget{esclusas}{%
\subsubsection{Esclusas}\label{esclusas}}

Las esclusas son estructuras hidráulicas fundamentales en el Canal de
Panamá. Consisten en cámaras que se llenan o vacían de agua para mover
los barcos de un nivel a otro. El canal tiene tres juegos de esclusas:
Miraflores, Pedro Miguel y Gatún, cada uno con dos cámaras. Los barcos
entran en una cámara, las compuertas se cierran, y se bombea agua para
ajustar el nivel, permitiendo así el tránsito seguro entre los océanos.

\includegraphics{3I.png}

\hypertarget{compuertas}{%
\subsubsection{Compuertas}\label{compuertas}}

Las compuertas son puertas operadas por motores eléctricos que controlan
el flujo de agua en las esclusas. Cada compuerta pesa alrededor de 700
toneladas y mide 19 metros de ancho, diseñada para soportar la presión
del agua. Al cerrar las compuertas, se regula el nivel del agua en las
cámaras, facilitando el movimiento de los barcos.

\hypertarget{control-del-truxe1fico}{%
\subsubsection{Control del Tráfico}\label{control-del-truxe1fico}}

El control del tráfico es crucial para la eficiencia y seguridad del
canal. Los barcos deben reservar su paso con antelación y proporcionar
detalles sobre su carga y destino. Pilotos capacitados guían a los
barcos utilizando boyas y señales de luz. Este sistema garantiza una
navegación ordenada y segura a través del canal.

\hypertarget{importancia-del-canal}{%
\subsection{Importancia del Canal}\label{importancia-del-canal}}

\hypertarget{comercio-internacional}{%
\subsubsection{Comercio Internacional}\label{comercio-internacional}}

El canal es una de las rutas más importantes del comercio mundial,
manejando el 5\% del comercio global.

\hypertarget{economuxeda-de-panamuxe1}{%
\subsubsection{Economía de Panamá}\label{economuxeda-de-panamuxe1}}

Representa una fuente significativa de ingresos para Panamá,
contribuyendo aproximadamente al 10\% del PIB del país.

\hypertarget{reducciuxf3n-de-costos}{%
\subsubsection{Reducción de Costos}\label{reducciuxf3n-de-costos}}

Permite a los barcos evitar la larga travesía alrededor de América del
Sur, reduciendo considerablemente los costos de transporte.

\hypertarget{impacto-ambiental-del-canal-de-panamuxe1}{%
\subsection{Impacto Ambiental del Canal de
Panamá}\label{impacto-ambiental-del-canal-de-panamuxe1}}

\hypertarget{puxe9rdida-de-huxe1bitat}{%
\subsubsection{Pérdida de Hábitat}\label{puxe9rdida-de-huxe1bitat}}

La construcción del canal provocó la deforestación y la pérdida de
hábitats naturales.

\hypertarget{especies-invasoras}{%
\subsubsection{Especies Invasoras}\label{especies-invasoras}}

El canal ha facilitado la introducción de especies invasoras, afectando
negativamente el ecosistema local.

\hypertarget{calidad-del-agua}{%
\subsubsection{Calidad del Agua}\label{calidad-del-agua}}

El canal impacta la calidad del agua debido a los sedimentos y otros
materiales presentes.

\hypertarget{quuxe9-pasa-si-se-ve-afectado-el-funcionamiento-del-canal-de-panamuxe1}{%
\subsection{¿Qué pasa si se ve afectado el funcionamiento del canal de
Panamá?}\label{quuxe9-pasa-si-se-ve-afectado-el-funcionamiento-del-canal-de-panamuxe1}}

En base a lo suscitado por ({``Por Qué El {Canal} de {Panamá} Se Está
Quedando Sin Agua (y No Tiene Nada Que Ver Con La Crisis Del
Coronavirus)''} n.d.) ,(Eavis and Angarita 2023) y por ({``Cambio
Climático y El {Canal} de {Panamá}''} 2023) desarrollamos las posibles
consecuencias de un mal funcionamiento del canal de Panamá.

Por más de un siglo, el Canal de Panamá ha sido la ruta más corta entre
los dos mayores océanos del mundo.

Por él pasa casi el 6\% del comercio mundial: cada año, más de 12.000
barcos lo cruzan de un lado a otro para llevar sus mercancías o
pasajeros por más de 140 rutas a más de 160 países.

Sin embargo, el funcionamiento del Canal de Panamá enfrenta serias
amenazas debido al cambio climático y otros factores. La sequía, por
ejemplo, ha reducido significativamente el nivel de agua en el Lago
Gatún, un componente crítico del canal.

Si la situación del Canal de Panamá empeora,habrían mayores
restricciones acerca del paso y el peso de los buques y las
consecuencias serían severas tanto a nivel local como global. Se prevé
que las restricciones en el tránsito de barcos se incrementen, lo que
llevaría a una reducción significativa en el volumen de mercancías que
pueden pasar por el canal. Esto no solo afectaría a las grandes
navieras, sino también a pequeñas y medianas empresas que dependen del
comercio internacional para sus operaciones diarias. Las rutas
alternativas, como el paso por el Cabo de Hornos o el Canal de Suez, se
volverían necesarias, incrementando los costos de transporte y los
tiempos de entrega. Las rutas más largas significarían un mayor consumo
de combustible, lo que a su vez elevaría los precios de los bienes y
servicios en los mercados globales, afectando a los consumidores
finales.

Además, las cadenas de suministro globales se verían gravemente
afectadas, especialmente para las industrias que dependen de entregas
just-in-time, como la automotriz, la tecnología y la manufactura. Los
retrasos en la entrega de componentes críticos podrían llevar a paradas
en la producción, pérdidas económicas y desempleo. La incertidumbre y la
falta de previsibilidad en los tiempos de tránsito también podrían
obligar a las empresas a mantener mayores inventarios, incrementando sus
costos operativos.

Panamá, por su parte, enfrentaría una disminución drástica en los
ingresos derivados del canal, lo que afectaría su economía y podría
llevar a recortes en servicios públicos y proyectos de infraestructura
esenciales. La pérdida de ingresos podría provocar una desaceleración
económica en el país, afectando sectores clave como la construcción, el
turismo y los servicios financieros. Además, la dependencia de rutas
alternativas incrementaría la huella de carbono del transporte marítimo,
agravando el cambio climático y creando un ciclo perjudicial para la
sostenibilidad global. La situación también podría generar tensiones
geopolíticas, ya que los países y las empresas buscan asegurar rutas
comerciales seguras y eficientes.

La situación requiere una acción inmediata y soluciones innovadoras para
mitigar estos riesgos y asegurar la continuidad del comercio
internacional. La implementación de tecnologías avanzadas, como gemelos
digitales y sistemas de gestión del agua más eficientes, podría mejorar
la resiliencia del canal. Asimismo, la cooperación internacional y la
inversión en infraestructuras críticas alternativas podrían ayudar a
diversificar las rutas de transporte y reducir la dependencia excesiva
de un solo canal. Solo a través de un enfoque integral y coordinado se
podrán enfrentar los desafíos y garantizar la estabilidad del comercio
global en el futuro.

\hypertarget{conclusiuxf3n}{%
\section{Conclusión}\label{conclusiuxf3n}}

En resumen, cualquier interrupción o disminución en la capacidad
operativa del Canal de Panamá tiene efectos en cadena que afectan tanto
al comercio internacional como a la economía de Panamá. La
implementación de soluciones innovadoras, como los gemelos digitales, es
crucial para mejorar la resiliencia y eficiencia de esta infraestructura
crítica frente a los desafíos actuales y futuros.

\hypertarget{refs}{}
\begin{CSLReferences}{1}{0}
\leavevmode\vadjust pre{\hypertarget{ref-noauthor_cambio_2023}{}}%
{``Cambio Climático y El {Canal} de {Panamá}.''} 2023.
\url{https://themys.sid.uncu.edu.ar/index.php/2023/10/10/cambio-climatico-y-el-canal-de-panama/}.

\leavevmode\vadjust pre{\hypertarget{ref-noauthor_critical_nodate}{}}%
{``Critical {Infrastructure} {Sectors} {\textbar} {CISA}.''} n.d.
Accessed August 5, 2024.
\url{https://www.cisa.gov/topics/critical-infrastructure-security-and-resilience/critical-infrastructure-sectors}.

\leavevmode\vadjust pre{\hypertarget{ref-noauthor_descubre_2023}{}}%
{``Descubre {Cómo} {Funciona} El {Canal} de {Panamá}: {Historia} y
{Funcionamiento}.''} 2023.
\url{https://panama50.com/como-funciona-el-canal-de-panama-una-guia-completa/}.

\leavevmode\vadjust pre{\hypertarget{ref-eavis_sequiafecta_2023}{}}%
Eavis, Peter, and Nathalia Angarita. 2023. {``Una Sequía Afecta Al
{Canal} de {Panamá}. {Y} El Comercio Internacional Lo Resiente.''}
\emph{The New York Times}, November.
\url{https://www.nytimes.com/es/2023/11/01/espanol/sequia-canal-de-panama-comercio.html}.

\leavevmode\vadjust pre{\hypertarget{ref-noauthor_infraestructuras_nodate}{}}%
{``Infraestructuras Críticas: Definición, Planes, Riesgos, Amenazas y
Legislación.''} n.d. \emph{LISA Institute}. Accessed August 5, 2024.
\url{https://www.lisainstitute.com/blogs/blog/infraestructuras-criticas}.

\leavevmode\vadjust pre{\hypertarget{ref-noauthor_por_nodate}{}}%
{``Por Qué El {Canal} de {Panamá} Se Está Quedando Sin Agua (y No Tiene
Nada Que Ver Con La Crisis Del Coronavirus).''} n.d. \emph{BBC News
Mundo}. Accessed August 5, 2024.
\url{https://www.bbc.com/mundo/noticias-america-latina-51840165}.

\end{CSLReferences}

\bibliographystyle{unsrt}
\bibliography{inf.bib}


\end{document}
